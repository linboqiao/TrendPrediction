\documentclass{jmlr}
\jmlrvolume{1,}
\jmlryear{2017}
\usepackage[utf8]{inputenc}
\usepackage{CJKutf8}
\usepackage{mypaper}

\begin{document}
\begin{CJK*}{UTF8}{gbsn}

\title{学习笔记}
\author{乔林波 \Email{linboqiao@gmail.com}\\
徐晓明 \Email{xxx}\\
刘栋 \Email{xxx}\\
\addr 团队:木头\\
}
\date{}
\maketitle

\section{赛题描述}

\subsection{简介}
对股票价格趋势的预测是金融领域极为复杂和极为关键的问题,有效市场假说认为股票价格趋势不可能被预测,然而真实市场由于各种因素的存在并不完全有效,这对于股票市场而言相当于一种“错误”。这里我们为参赛者提供了大规模的股票历史数据,从而可以通过集合大家的智慧来纠正股票市场的这些“错误”。

\subsection{数据}
数据集包括训练数据集和测试数据集两部分。

训练数据集是一个以逗号分隔的文本文件(csv),
其中:id列为数据唯一标识编码,feature列为原始数据经过变换之后得到的特征,weight列为样本重要性,label列为待预测二分类标签,group列为样本所属分组编号,era列为时间区间编号(取值1-20为时间顺序)。

测试数据集是一个以逗号分隔的文本文件(csv),
其中:id列为数据唯一标识编码,feature列为原始数据经过变换之后得到的特征。测试数据集不包括weight列、label列和era列。

\subsection{评价标准}
虚拟股票趋势预测比赛的评价指标类比一般二分类问题的评价方式,将最终的logloss值作为最终选手排名的依据,logloss的计算方法如下:

\be
l(y_p;y_t)=-\sum_{i=1}^N \left( w^i *\left( y_t^i * \ln(y_p^i)+(1-y_t^i) * \ln(1-y_p^i) \right) \right)
\ee
其中, $y_t=\{y_t^0,\cdots,y_t^i,\cdots,y_t^N\}$ 是样本的真实标签, $y_p=\{y_p^0,\cdots,y_p^i,\cdots,y_p^N\}$ 是样本预测为正类的概率,$w_i$ 是第$i$个样本的样本权重,$N$是测试集样本数量。


\section{求解}

\subsection{使用简单二分类求解}
將原问题视为一个二分类问题,

\subsection{note}
对于交叉验证,建议按照训练数据era列随机抽取一个或若干个era进行交叉验证,而不是在全部训练样本上进行随机采样进行交叉验证,因为后者会导致严重的过拟合问题,这也是我们加入了era列的主要目的。




\end{CJK*}
\end{document}
